\documentclass{article}
\usepackage{listings}
\usepackage{xcolor}

\lstset{
    basicstyle=\ttfamily,
    keywordstyle=\color{blue},
    stringstyle=\color{red},
    commentstyle=\color{green},
    showstringspaces=false
}

\title{Software Stack for Video Analysis Project}
\author{}
\date{}

\begin{document}

\maketitle

\section{Introduction}
This document provides an overview of the software stack used in the video analysis project. It lists the different libraries, explains their purposes, and provides basic usage examples.

\section{Libraries}

\subsection{OpenCV}
\textbf{Description}: OpenCV (Open Source Computer Vision Library) is a library of programming functions mainly aimed at real-time computer vision.

\textbf{Installation}:
\begin{lstlisting}[language=bash]
conda install -c conda-forge opencv
\end{lstlisting}

\textbf{Basic Usage}:
\begin{lstlisting}[language=python]
import cv2

# Load an image
image = cv2.imread('path_to_image.jpg')

# Display the image
cv2.imshow('Image', image)
cv2.waitKey(0)
cv2.destroyAllWindows()
\end{lstlisting}

\subsection{TensorFlow}
\textbf{Description}: TensorFlow is an open-source machine learning library developed by Google. It is used for building and deploying machine learning models.

\textbf{Installation}:
\begin{lstlisting}[language=bash]
conda install tensorflow=2.4.1
\end{lstlisting}

\textbf{Basic Usage}:
\begin{lstlisting}[language=python]
import tensorflow as tf

# Define a constant
hello = tf.constant('Hello, TensorFlow!')

# Start a TensorFlow session
sess = tf.compat.v1.Session()

# Run the session
print(sess.run(hello))
\end{lstlisting}

\subsection{NumPy}
\textbf{Description}: NumPy is a fundamental package for scientific computing with Python. It provides support for arrays, matrices, and many mathematical functions.

\textbf{Installation}:
\begin{lstlisting}[language=bash]
conda install numpy=1.19.5
\end{lstlisting}

\textbf{Basic Usage}:
\begin{lstlisting}[language=python]
import numpy as np

# Create an array
array = np.array([1, 2, 3, 4, 5])

# Perform an operation
print(np.mean(array))
\end{lstlisting}

\subsection{Pandas}
\textbf{Description}: Pandas is a library providing high-performance, easy-to-use data structures and data analysis tools for Python.

\textbf{Installation}:
\begin{lstlisting}[language=bash]
conda install pandas=1.2.3
\end{lstlisting}

\textbf{Basic Usage}:
\begin{lstlisting}[language=python]
import pandas as pd

# Create a DataFrame
data = {'Name': ['Alice', 'Bob', 'Charlie'], 'Age': [25, 30, 35]}
df = pd.DataFrame(data)

# Display the DataFrame
print(df)
\end{lstlisting}

\subsection{Scikit-learn}
\textbf{Description}: Scikit-learn is a machine learning library for Python that provides simple and efficient tools for data mining and data analysis.

\textbf{Installation}:
\begin{lstlisting}[language=bash]
conda install scikit-learn
\end{lstlisting}

\textbf{Basic Usage}:
\begin{lstlisting}[language=python]
from sklearn.datasets import load_iris
from sklearn.model_selection import train_test_split
from sklearn.ensemble import RandomForestClassifier
from sklearn.metrics import accuracy_score

# Load dataset
iris = load_iris()
X = iris.data
y = iris.target

# Split the dataset
X_train, X_test, y_train, y_test = train_test_split(X, y, test_size=0.2, random_state=42)

# Initialize and train the model
model = RandomForestClassifier()
model.fit(X_train, y_train)

# Make predictions
y_pred = model.predict(X_test)

# Evaluate the model
accuracy = accuracy_score(y_test, y_pred)
print(f'Accuracy: {accuracy:.2f}')
\end{lstlisting}

\subsection{PyTorch}
\textbf{Description}: PyTorch is an open-source machine learning library based on the Torch library, used for applications such as computer vision and natural language processing.

\textbf{Installation}:
\begin{lstlisting}[language=bash]
conda install pytorch torchvision torchaudio -c pytorch
\end{lstlisting}

\textbf{Basic Usage}:
\begin{lstlisting}[language=python]
import torch

# Create a tensor
tensor = torch.tensor([1, 2, 3, 4, 5])

# Perform an operation
print(torch.mean(tensor.float()))
\end{lstlisting}

\subsection{SQLite}
\textbf{Description}: SQLite is a C-language library that provides a lightweight, disk-based database. It is used for local/client storage in application software.

\textbf{Installation}:
\begin{lstlisting}[language=bash]
conda install sqlite
\end{lstlisting}

\textbf{Basic Usage}:
\begin{lstlisting}[language=python]
import sqlite3

# Connect to SQLite database
conn = sqlite3.connect('example.db')
cursor = conn.cursor()

# Create a table
cursor.execute('''CREATE TABLE users (id INTEGER PRIMARY KEY, name TEXT, age INTEGER)''')

# Insert data
cursor.execute('''INSERT INTO users (name, age) VALUES ('Alice', 25)''')

# Commit and close
conn.commit()
conn.close()
\end{lstlisting}

\subsection{FFmpeg}
\textbf{Description}: FFmpeg is a multimedia framework used to decode, encode, transcode, mux, demux, stream, filter, and play almost anything that humans and machines have created.

\textbf{Installation}:
\begin{lstlisting}[language=bash]
conda install -c conda-forge ffmpeg
\end{lstlisting}

\textbf{Basic Usage}:
\begin{lstlisting}[language=python]
import subprocess

def check_ffmpeg():
    try:
        result = subprocess.run(['ffmpeg', '-version'], stdout=subprocess.PIPE, stderr=subprocess.PIPE)
        print("FFmpeg version:\n", result.stdout.decode('utf-8').split('\n')[0])
    except FileNotFoundError:
        print("FFmpeg is not installed or not found in PATH.")

check_ffmpeg()
\end{lstlisting}

\section{Summary}
This document provides an overview of the essential libraries used in the video analysis project, along with installation commands and basic usage examples. These libraries include OpenCV, TensorFlow, NumPy, Pandas, Scikit-learn, PyTorch, SQLite, and FFmpeg.

\end{document}
